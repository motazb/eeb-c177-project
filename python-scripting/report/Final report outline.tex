%preamble (before the document)
\documentclass{article}
%article is the typical document type people use, u can look up other ones.

\usepackage[utf8]{inputenc}
%tells it the type of characters we will be using within this document.
% if u download the font package: \usepackage{times}
\usepackage{graphicx}
\graphicspath{ {images/} }

\usepackage{color} %allows u to put colors

\title{My Final Project}
\author{Moataz Basyouni}
\date {February 2020}
%you can also use \today and it will put today's date automatically.


\begin{document}

\maketitle

\section{Abstract}
Talk about the program and the data and what was found out about the data after running the program on it.
\newpage

\tableofcontents
\listoffigures
\newpage

\section{Introduction}
This program uses data from a collection of product-related injury reports in the US from the National Electronic Injury Surveillance System (NEISS), which is a national probability sample of hospitals in the U.S. and its territories. The data includes patient information from each NEISS hospital for every emergency visit involving an injury associated with consumer products. The types of products involved in injuries include: floors, stairs, beds, basketball, bicycles, football, chairs, knives, bathtubs/showers, ceiling/walls.

This program is important because it gives a perspective on the frequencies of injuries caused by different products and the demographics of patients who were injured by them. By determining the demographics of those injured by these products, further studies can be conducted to find out why they are more likely to be injured by them, and how to prevent that from continuing to happen.

Talk about the purpose of the program and what it does. Talk about who can use it and its importance.\\\\\\


\section{Methods}
Talk about the code used and how I made it. Talk about how its run by Python through the terminal.\\

\begin{verbatim}
#!/usr/bin/env python

def openfile(fname):
    import numpy
    import pandas as pd
    tmp_data = pd.read_csv(fname)
    data = tmp_data.to_numpy()
openfile('patientdata.csv')
#defines a function openfile so that whenever the function
#is called with the name of a file, it will open that file

def maxage(fname):
    import numpy
    import pandas as pd
    tmp_data = pd.read_csv(fname)
    data = tmp_data.to_numpy()
    age = data[:,4]
    age_max = numpy.max(age)
    print("The oldest patient is ", age_max, " years old.")
maxage('patientdata.csv')
#this function looks through the fourth column in the file
#and finds the max value. Sifting through the fourth column
#which includes the ages, and finding the max value will help
#find the oldest person in the list

print("\n")

def display(source, numlines = 3):
    print("Here is a preview of the file:")
    with open(source) as f:
        for i, line in enumerate(f):
            print(line.strip())
            if i == numlines:
                break
display('patientdata.csv')
#This function gives a preview of the file with the number of lines
#specified. If number of lines isn't specified, it will show a max
#of 3 lines.

print("\n")

def displaydict(source, numlines = 3):
    print("Here is a preview of the file in a dictionary")
    import csv
    with open(source) as f:
        reader = csv.DictReader(f)
        for i, row in enumerate(reader):
            print(dict(row))
            if i == numlines:
                break
displaydict('patientdata.csv')
#This gives a perview of the file in a dictionary.

print("\n")

def importgenderdata(source, output, gender):
    print("Here is a preview of the female patients only")
    import csv
    assert gender != "Female" or "Male", "Please ensure first letter of gender is capitalized"
    with open (source) as fr:
        reader = csv.DictReader(fr)
        header = reader.fieldnames
        with open (output, "w") as fw:
            writer = csv.DictWriter(fw, fieldnames = header, delimiter = ",")
            for row in reader:
                if row["sex"] == gender:
                    writer.writerow(row)
importgenderdata('patientdata.csv', 'femaledata.csv', 'Female')
with open('patientdata.csv') as f:
    for i, line in enumerate(f):
        print(line.strip())
        if i == 3:
            break
#This function imports the female data only into a new file,
#it then previews some of it as done before
print("\n")

def hist(source):
    import numpy
    import pandas as pd
    tmp_data = pd.read_csv(source)
    data = tmp_data.to_numpy()
    import matplotlib.pyplot as plt
    ages = data[:,4]
    x = [ages]
    plt.style.use('ggplot')
    plt.hist(x, bins = 10)
hist('patientdata.csv')
#This functions uses data from column 4 (ages), and builds
#a histogram with it. It can be used as a visual representation
#of all the ages of the patients

print("\n")

def julydates():
    import csv
    import re
    with open("patientdata.csv") as fr:
        dictread = csv.DictReader(fr, delimiter = ',')
        header = dictread.fieldnames
        dates = []
        for row in dictread:
            dates.append(row['treatmentDate'])
    print('Here are the treatment dates which occured in July')
    date_regex = re.compile(r'7/\d*/\d{4}')
    uniqdates = set(dates)
    for date in uniqdates:
        if re.match(date_regex, date):
            print(date)

julydates()
#This function displays only the treatment dates which occured in July of 2015

\end{verbatim}


\section{Results}
Talk about the outputs of my program after inputting my dataset. \\\\

\subsection{Figures}
Insert the histograms and discuss them\\\\

\section{Discussion}
Explain the results that the program produced and their purpose and how they can be used.

\section{Conclusion}
summarize every section\\\\

\section{References}
Reference stackoverflow and the other sources used\\\\

\end{document}
